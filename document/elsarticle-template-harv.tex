%% 
%% Copyright 2007-2024 Elsevier Ltd
%% 
%% This file is part of the 'Elsarticle Bundle'.
%% ---------------------------------------------
%% 
%% It may be distributed under the conditions of the LaTeX Project Public
%% License, either version 1.3 of this license or (at your option) any
%% later version.  The latest version of this license is in
%%    http://www.latex-project.org/lppl.txt
%% and version 1.3 or later is part of all distributions of LaTeX
%% version 1999/12/01 or later.
%% 
%% The list of all files belonging to the 'Elsarticle Bundle' is
%% given in the file `manifest.txt'.
%% 
%% Template article for Elsevier's document class `elsarticle'
%% with harvard style bibliographic references

\documentclass[preprint,12pt,authoryear]{elsarticle}
\usepackage{graphicx}      % include this line if your document contains figures
\usepackage{hyperref}
\usepackage{natbib}        % required for bibliography
%\usepackage{amsfonts}
\usepackage{tabularx}
\usepackage{caption}
%\usepackage{amssymb}
\usepackage{amsmath}
\usepackage{mathtools}
\usepackage{siunitx}
\usepackage{subfigure}
\usepackage{algorithm}
%\usepackage{algpseudocode}
\usepackage{booktabs}
\usepackage{url}
\usepackage{cite}
\usepackage{amsmath,amssymb,amsfonts}
\usepackage{algorithmic}
\usepackage{graphicx}
\usepackage{textcomp}
\usepackage{xcolor}
\usepackage{float}
\def\BibTeX{{\rm B\kern-.05em{\sc i\kern-.025em b}\kern-.08em
    T\kern-.1667em\lower.7ex\hbox{E}\kern-.125emX}}

\DeclareMathOperator{\argmin}{arg\;min}
\DeclareMathOperator{\vol}{vol}
\DeclareMathOperator{\diag}{diag}
\newcommand{\ui}[2]{#1_{\text{#2}}}
\newcommand{\uis}[2]{#1^{\text{#2}}}
\newcommand{\lrp}[1]{\left( #1 \right)}
\newcommand{\uib}[2]{{\bf #1}_{\text #2}}
\newcommand{\Ts}{\ui{T}{s}}

%% NN controller commands
\newcommand{\cnn}{\ui{\mathcal{C}}{NN}}
\newcommand{\fddpg}{\ui{f}{E}}
\newcommand{\uddpg}{\ui{u}{E}}
\newcommand{\actor}{\ui{f}{ACT}}
\newcommand{\critic}{\ui{f}{CRIT}}
\newcommand{\thx}{\ui{\theta}{x}}
\newcommand{\thf}{\ui{\theta}{f}}
\newcommand{\ucorr}{\widetilde{u}}
\newcommand{\unn}{\ui{u}{nn}}
\newcommand{\uact}{\ui{u}{ACT}}
\newcommand{\corr}{\ui{\mathcal{C}}{C}}
\newcommand{\todo}[1]{{{\color{red} TODO: #1	}} }
\renewcommand{\mark}[1]{{{\color{green} #1	}} }
%\newcommand{\e}[1]{\cdot 10^{#1}}
%%-------------------------------------------------------
%% MACRO for tikz picture. DO NOT CHANGE ANYTHING !!!!!
\usepackage{tikz}
\usepackage{pgfkeys}
\usepackage{calc}
\usepackage{pgfplots}
\usetikzlibrary{arrows}
\usetikzlibrary{calc}
\usepgflibrary{arrows}
\usetikzlibrary{positioning}
\usetikzlibrary{intersections}
\pgfplotsset{compat=newest}
\usetikzlibrary{shapes.geometric}
\usetikzlibrary{decorations.pathreplacing}
\usetikzlibrary{patterns,decorations.pathmorphing,decorations.markings}
\usetikzlibrary{external}
\usetikzlibrary{plotmarks}
\usetikzlibrary{arrows.meta}
\usepgfplotslibrary{patchplots}
\usepackage{grffile}
\usepgfplotslibrary{fillbetween}
\usepackage{xfrac}
%\tikzexternalize
\newcommand{\R}{\mathbb{R}}
\newcommand{\includetikz}[1]{%
\tikzifexternalizing{%
\def\DOIT{1}%
}{%
\IfFileExists{#1.pdf}{%
\includegraphics[scale=1]{#1.pdf}%
\def\DOIT{0}%
}{%
\def\DOIT{1}%
}%
}%
%
\if1\DOIT
%	\tikzsetnextfilename{mypic_#1}%
\tikzsetnextfilename{#1}
%   \filemodCmp{#1.tikz}{external/#1.log}%
%  {\tikzset{external/force remake=true}\input{#1.tikz}}
\input{#1.tikz}
\fi
}

% defines lengths for figure plotting
\newlength\figureheight
\newlength\figurewidth
%%  End of TikZ Macros
%%-------------------------------------------------------
\begin{document}

\begin{frontmatter}

%% Title, authors and addresses

%% use the tnoteref command within \title for footnotes;
%% use the tnotetext command for theassociated footnote;
%% use the fnref command within \author or \affiliation for footnotes;
%% use the fntext command for theassociated footnote;
%% use the corref command within \author for corresponding author footnotes;
%% use the cortext command for theassociated footnote;
%% use the ead command for the email address,
%% and the form \ead[url] for the home page:
%% \title{Title\tnoteref{label1}}
%% \tnotetext[label1]{}
%% \author{Name\corref{cor1}\fnref{label2}}
%% \ead{email address}
%% \ead[url]{home page}
%% \fntext[label2]{}
%% \cortext[cor1]{}
%% \affiliation{organization={},
%%            addressline={}, 
%%            city={},
%%            postcode={}, 
%%            state={},
%%            country={}}
%% \fntext[label3]{}

\title{The Beggining of Control Revolution: Ofset-free Koopman MPC} %% Article title

%% use optional labels to link authors explicitly to addresses:
%% \author[label1,label2]{}
%% \affiliation[label1]{organization={},
%%             addressline={},
%%             city={},
%%             postcode={},
%%             state={},
%%             country={}}
%%
%% \affiliation[label2]{organization={},
%%             addressline={},
%%             city={},
%%             postcode={},
%%             state={},
%%             country={}}

\author{Patrik Valábek} %% Author name

%% Author affiliation
\affiliation{organization={STU},%Department and Organization
            addressline={Redlinskeho 9}, 
            city={Bratislava},
            country={Slovakia}}

%% Abstract
\begin{abstract}
%% Text of abstract

\end{abstract}

%%Graphical abstract
%\begin{graphicalabstract}
%\includegraphics{grabs}
%\end{graphicalabstract}

%%Research highlights
\begin{highlights}
\item Research highlight 1
\item Research highlight 2
\end{highlights}

%% Keywords
\begin{keyword}
%% keywords here, in the form: keyword \sep keyword

%% PACS codes here, in the form: \PACS code \sep code

%% MSC codes here, in the form: \MSC code \sep code
%% or \MSC[2008] code \sep code (2000 is the default)

\end{keyword}

\end{frontmatter}

%% Add \usepackage{lineno} before \begin{document} and uncomment 
%% following line to enable line numbers
%% \linenumbers

%% main text
%%
\section{Introduction}
\label{sec:intro}

\section{Preliminaries and Notation}
\label{sec:Preliminaries}

\section{Results}
\label{sec:Results}


%\bibliographystyle{elsarticle-num-names}  % Elsevier's numbered citation style
%\bibliography{bibfile}
 % Reference your .bib file (without .bib extension)
\end{document}
 
\endinput
%%
%% End of file `elsarticle-template-harv.tex'.


